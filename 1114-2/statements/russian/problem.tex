\begin{problem}{МКАД}{стандартный ввод}{стандартный вывод}{1 секунда}{16 мегабайт}

Длина Московской кольцевой автомобильной дороги ---$109$ километров. Байкер Вася стартует с первого километра МКАД и едет со скоростью $V$ километров в час. На какой отметке он остановится через $T$ часов?

\InputFile
Первая строка содержит два целых числа $V$ и $T$ -- скорость (км/ч) и время поездки в часах соответственно. Числа разделены пробелом. Если $V>0$, то Вася движется в положительном направлении по МКАД, если же значение $V<0$, то в отрицательном. Ограничения: $|V| \leq 1000$, $0 \leq T \leq 1000$.

\OutputFile
Выведите целое число от $1$ до $109$ -- километр МКАД, на котором остановится Вася.

\Examples

\begin{example}
\exmpfile{example.01}{example.01.a}%
\exmpfile{example.02}{example.02.a}%
\end{example}

\end{problem}

