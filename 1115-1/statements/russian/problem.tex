\begin{problem}{Дележка яблок}{стандартный ввод}{стандартный вывод}{1 секунда}{16 мегабайт}

$N$ школьников желают разделить $K$ яблок между собой. Они рассматривают два способа дележа:

1.разделить яблоки поровну так, чтобы каждому досталось максимальное количество яблок, при этом оставшиеся яблоки можно положить в корзину;

2.разделить все яблоки так, чтобы количество яблок, доставшихся любым двум школьникам, отличалось бы не более, чем на 1. В этом случае могут обидеться те из них, кому достанется яблок меньше, чем другим.

\InputFile
Входной файл INPUT.TXT содержит натуральные числа $N$ и $K$ -- количество школьников и яблок соответственно $N, K \leq 10^9$.

\OutputFile
В выходной файл OUTPUT.TXT выведите три целых числа через пробел:

a.количество яблок, которые достанутся всем школьником при первом способе дележа;

b.количество яблок, которые окажутся в корзине при первом способе дележа;

c.количество обиженных школьников во втором случае дележа.

\Examples

\begin{example}
\exmpfile{example.01}{example.01.a}%
\exmpfile{example.02}{example.02.a}%
\end{example}

\end{problem}

