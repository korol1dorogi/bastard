\begin{problem}{<<Гулливер>>}{стандартный ввод}{стандартный вывод}{1 секунда}{16 мегабайт}

Из книги Джонатана Свифта мы знаем, что тот Гулливер посетил страну <<Лилипутию>>, где живут лилипуты, окруженные вещами, животными и заводами небольшого размера. Сначала лилипуты боялись Гулливера, но позже они поняли, что такое соседство приносит им большую выгоду, и они стали помогать ему. Например, лилипуты делали кровать для Гулливера из своих маленьких матрацев, сшитых вместе. Лилипутам были известны размеры Гулливера. Довольно быстро они смогли просчитать количество матрацев, необходимых для шитья большого матраца. Но у них постоянно возникали сложности с их небольшим ростом и стеля постель, они иногда не могли сшить достаточно толстый матрац.

\InputFile
Входные данные
На вход поступает два целых числа, которые разделены пробелом: $K$ -- коэффициент, отражающий во сколько раз Гулливер больше лилипутов, и $M$ -- количество слоев матрацев $2 \le K, M \le 100$.

\OutputFile
Выходные данные
Выведите количество матрацев лилипутов, необходимых для построения матраца для Гулливера.

\Example

\begin{example}
\exmpfile{example.01}{example.01.a}%
\end{example}

\end{problem}

